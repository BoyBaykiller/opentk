\documentclass[a4paper, 12pt]{article}

\usepackage[margin=1in]{geometry}
\usepackage{hyperref}
\usepackage[acronym,automake]{glossaries}

\hypersetup{colorlinks=true}

\title{
    Platform Abstraction Layer (PAL) 2.0\\
    {\large Specification Draft}
}
\author{OpenTK}

\makeglossaries

\newacronym{pal}{PAL}{Platform Abstraction Layer}


\newcommand{\tkissue}[1]{\href{https://github.com/opentk/opentk/issues/#1}{issue \##1}} 

\begin{document}
\maketitle

\pagenumbering{roman}
\section*{Abstract}
Over the years the OpenTK project has use several different methods for
abstracting platform APIs (such as but not limited to Win32 and X11). The latest
method, the GLFW library has proven to be troublesome and lack the cohesion of
previous \gls{pal} of the OpenTK 3.0 Version. This document aims to redesign the
original \gls{pal} into C-like interface from which object oriented types may be
derived for all platforms which may be partially implemented depending on the
executing system.

\tableofcontents
\clearpage

\pagenumbering{arabic}
\setcounter{page}{1}

\section{Introduction}

\subsection{The Problem At Hand}
With the somewhat recent changes made in OpenTK 4.0 and later, the public API
has been found to be lacking the features the legacy OpenTK had. Many users have
been forced to call GLFW directly, defeating the purpose of OpenTK for providing
a type safe, object oriented platform abraction layer. An ongoing example of
this is the inability to enumerate the video mode capability of connected
displays. (\tkissue{1221}) 

Another problem with GLFW is that it is incapable of creating OpenGL contexts
with a version less than or equal to version 3.2. There is enough demand that
an outstanding issue exists, \tkissue{1040}. This is unacceptable, however,
this may only be fixed by updating GLFW with this capability, which brings me to
the final point.

GLFW is a needless dependency. Even though the project is open source, if a bug
or issue surfaces which may only be fixed in GLFW, a fix may take a long time to
be approved or never be approved even if implemented by OpenTK. This would
require distributing our own fork of GLFW which is more trouble for maintaining
the project. Since it is a 3rd party dependency, sometimes issues rise up with
GLFW not being present at the binary directory, even if it is explicitly
provided by the nuget package. (see \tkissue{1228})

It is in our best interest to keep OpenTK freestanding from 3rd party
abstraction layers. However, this isn't mutually exclusive to the idea of
supporting dependents from implementing such abstraction layers into OpenTK.


\clearpage

\printglossaries
\addcontentsline{toc}{section}{Glossary}

\end{document}